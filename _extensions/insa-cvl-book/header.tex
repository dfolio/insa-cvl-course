%
\usepackage{booktabs}
\usepackage{colortbl}
\usepackage{relsize}
\usepackage[version=4]{mhchem}
\usepackage[]{fontawesome5}
\usepackage{physics}
\DeclareMathOperator{\arcosh}{arcosh}
\DeclareMathOperator{\arsinh}{arsinh}
\DeclareDocumentCommand\acosh{}{\trigbraces{\arcosh}}
\DeclareDocumentCommand\asinh{}{\trigbraces{\arsinh}}
\usepackage{siunitx}
\sisetup{per-mode = symbol}
% url %%%%%%%%%%%%%%%%%%%%%%%%%%%%%%%%%%%%%%%%%%%%%%%%%%%%%%%%%%%%%%%%%%%%%%%%%%
\usepackage{url}
\renewcommand{\UrlFont}{\ttfamily\smaller}
\DeclareUrlCommand\email{\urlstyle{tt}\smaller}
% columun %%%%%%%%%%%%%%%%%%%%%%%%%%%%%%%%%%%%%%%%%%%%%%%%%%%%%%%%%%%%%%%%%%%%%%
\newenvironment{cols}[1][]{}{}
\let\columns\cols
\let\endcolumns\endcols
\newenvironment{col}[2][]{\begin{minipage}[#1]{#2}\ignorespaces}{%
\end{minipage}
\ifhmode\unskip\fi
\aftergroup\useignorespacesandallpars}

\let\column\col
\let\endcolumn\endcol

\def\useignorespacesandallpars#1\ignorespaces\fi{%
#1\fi\ignorespacesandallpars}

\makeatletter
\def\ignorespacesandallpars{%
  \@ifnextchar\par
    {\expandafter\ignorespacesandallpars\@gobble}%
    {\hfill}%
}

% itemize/enumerate%%%%%%%%%%%%%%%%%%%%%%%%%%%%%%%%%%%%%%%%%%%%%%%%%%%%%%%%%%%%%
\usepackage{enumitem}
\setlist{noitemsep,nosep,leftmargin=*}
\setlist[1]{labelindent=\parindent,topsep=-.5em,partopsep=0pt} % < Usually a good idea
\setlist[itemize,1]{label={\relsize{-2}\faIcon[regular]{square}}}
\setlist[itemize,2]{label={\relsize{-3}\faIcon[regular]{circle}}}
\setlist[itemize,3]{label={\relsize{-2}\faIcon[solid]{chevron-right}}}
\setlist[description]{font=\smaller,topsep=0\baselineskip,parsep=0.5\baselineskip,}
\setlist[description,1]{style=nextline, leftmargin=1em}

% Theorem %%%%%%%%%%%%%%%%%%%%%%%%%%%%%%%%%%%%%%%%%%%%%%%%%%%%%%%%%%%%%%%%%%%%%%
\usepackage{amsthm}
\newtheoremstyle{plain}          % name
    {.5em}                       % Space above
    {.5em}                       % Space below
    {\itshape}                   % Body font
    {}                           % Indent amount
    {\sffamily\scshape\bfseries} % Theorem head font
    {.}                          % Punctuation after theorem head
    {.5em}                       % Space after theorem head
    {}  % Theorem head spec (can be left empty, meaning ‘normal’)
\newtheoremstyle{definition} % name
    {.25em}                      % Space above
    {-.5em}                      % Space below
    {}                           % Body font
    {}                           % Indent amount
    {\sffamily\scshape}          % Theorem head font
    {.}                          % Punctuation after theorem head
    {.5em}                       % Space after theorem head
    {}  % Theorem head spec (can be left empty, meaning ‘normal’)
\newtheoremstyle{remark} % name
    {.25em}                      % Space above
    {-.5em}                      % Space below
    {\color{black!80}}           % Body font
    {}                           % Indent amount
    {\smaller\sffamily\color{black!80}}          % Theorem head font
    {:}                          % Punctuation after theorem head
    {.5em}                       % Space after theorem head
    {}  % Theorem head spec (can be left empty, meaning ‘normal’)
% part  %%%%%%%%%%%%%%%%%%%%%%%%%%%%%%%%%%%%%%%%%%%%%%%%%%%%%%%%%%%%%%%%%%%%%
\renewcommand*{\addparttocentry}[2]{%
  \addtocentrydefault{part}{\partname\nobreakspace #1}{#2}%
}

\RedeclareSectionCommand[tocnumwidth=6em]{part}
% chapter  %%%%%%%%%%%%%%%%%%%%%%%%%%%%%%%%%%%%%%%%%%%%%%%%%%%%%%%%%%%%%%%%%%%%%
\RedeclareSectionCommand[tocnumwidth=2.5em]{chapter}
\RedeclareSectionCommand[tocnumwidth=3em]{section}
\renewcommand\thechapter{{\rmfamily\@Roman\c@chapter}}
%\RedeclareSectionCommand[style=section,indent=0pt,beforeskip=-2\baselineskip]{chapter}
\renewcommand*{\raggedchapter}{\raggedleft}
%\setkomafont{chapter}{\normalfont\sffamily\bfseries\scshape}
\addtokomafont{chapter}{\scshape}

\makeatletter
\renewcommand*{\chapterformat}{{\thechapter\autodot}}
\renewcommand*{\chapterheadmidvskip}{\enskip}
\renewcommand{\chapterlinesformat}[3]{%
\parbox{\dimexpr\linewidth-2\fboxrule-2\fboxsep}{\@hangfrom{#2}#3}%
}
\renewcommand{\chapterlineswithprefixformat}[3]{%
\raggedchapter
\interlinepenalty\@M
\parbox{\dimexpr\linewidth-2\fboxrule-2\fboxsep}{%
\raggedchapter
{#2~\textcolor{gray}{\rule[-0.5em]{4pt}{1.75em}}~}#3%
}
\par\vskip-0.5em plus0.25em
\rule{\linewidth}{2pt}%
\par\nobreak
}
\makeatother

\renewcommand{\chaptermark}[1]{\markboth{\sffamily\footnotesize Chap.\,\thechapter\ #1}{}}
\renewcommand{\sectionmark}[1]{\markright{\sffamily\footnotesize\thesection\ #1}}

\RedeclareSectionCommand[afterindent=false,beforeskip=0pt]{chapter}
\RedeclareSectionCommand[afterindent=false,beforeskip=0.5\baselineskip,afterskip=0.25\baselineskip]{section}
\RedeclareSectionCommand[afterindent=false,beforeskip=0.25\baselineskip,afterskip=0.15\baselineskip]{subsection}
\RedeclareSectionCommand[afterindent=false,beforeskip=0.2\baselineskip,afterskip=0.05\baselineskip]{subsubsection}
\RedeclareSectionCommand[afterindent=false,beforeskip=0.1\baselineskip,]{paragraph}
\RedeclareSectionCommand[afterindent=false,font=\normalfont\itshape,beforeskip=0.1\baselineskip]{subparagraph}

% Figure  %%%%%%%%%%%%%%%%%%%%%%%%%%%%%%%%%%%%%%%%%%%%%%%%%%%%%%%%%%%%%%%%%%%%%%
\renewcommand{\topfraction}{.85}
\renewcommand{\bottomfraction}{.7}
\renewcommand{\textfraction}{.15}
\renewcommand{\floatpagefraction}{.66}
\setcounter{topnumber}{3}
\setcounter{bottomnumber}{3}
\setcounter{totalnumber}{4}

\setlength\abovedisplayskip{0.5ex}%
\setlength\belowdisplayskip{0.5ex}%
\setlength\abovedisplayshortskip{0pt}%
\setlength\belowdisplayshortskip{3pt}%
% longtable patch %%%%%%%%%%%%%%%%%%%%%%%%%%%%%%%%%%%%%%%%%%%%%%%%%%%%%%%%%%%%%%
\makeatletter
\def\LT@start{%
  \let\LT@start\endgraf
  \endgraf\penalty\z@\vskip\LTpre\endgraf
   \ifdim \pagetotal<\pagegoal \else
      \dimen@=\pageshrink
      \advance \dimen@ 1sp %
      \kern\dimen@\penalty 9999\endgraf \kern-\dimen@
   \fi
  \dimen@\pagetotal
  \advance\dimen@ \ht\ifvoid\LT@firsthead\LT@head\else\LT@firsthead\fi
  \advance\dimen@ \dp\ifvoid\LT@firsthead\LT@head\else\LT@firsthead\fi
  \advance\dimen@ \ht\LT@foot
  \edef\LT@reset@vfuzz{\vfuzz\the\vfuzz\vbadness\the\vbadness\relax}%
  \vfuzz\maxdimen
  \vbadness\@M
  \setbox\tw@\copy\z@
  \setbox\tw@\vsplit\tw@ to \ht\@arstrutbox
  \setbox\tw@\vbox{\unvbox\tw@}%
  \LT@reset@vfuzz
  \advance\dimen@ \ht
        \ifdim\ht\@arstrutbox>\ht\tw@\@arstrutbox\else\tw@\fi
  \advance\dimen@\dp
        \ifdim\dp\@arstrutbox>\dp\tw@\@arstrutbox\else\tw@\fi
  \advance\dimen@ -\pagegoal
  \ifdim \dimen@>\z@
    \vfil\break
  \else
    \ifdim\pageshrink>\z@\pageshrink\z@\fi
  \fi
      \global\@colroom\@colht
  \ifvoid\LT@foot\else
    %\global\advance\vsize-\ht\LT@foot
    \global\advance\@colroom-\ht\LT@foot
    \dimen@\pagegoal\advance\dimen@-\ht\LT@foot\pagegoal\dimen@
    \maxdepth\z@
  \fi
  \ifvoid\LT@firsthead\copy\LT@head\else\box\LT@firsthead\fi\nobreak
  \output{\LT@output}}

\def\LT@output{%
  \ifnum\outputpenalty <-\@Mi
    \ifnum\outputpenalty > -\LT@end@pen
      \LT@err{floats and marginpars not allowed in a longtable}\@ehc
    \else
      \setbox\z@\vbox{\unvbox\@cclv}%
      \ifdim \ht\LT@lastfoot>\ht\LT@foot
        \dimen@\pagegoal
        \advance\dimen@\ht\LT@foot
        \advance\dimen@-\ht\LT@lastfoot
        \ifdim\dimen@<\ht\z@
          \setbox\@cclv\vbox{\unvbox\z@\copy\LT@foot\vss}%
          \@makecol
          \@outputpage
          \global\vsize\@colroom
          \setbox\z@\vbox{\box\LT@head}%
        \fi
      \fi
        \unvbox\z@\box\ifvoid\LT@lastfoot\LT@foot\else\LT@lastfoot\fi
    \fi
  \else
    \setbox\@cclv\vbox{\unvbox\@cclv\copy\LT@foot\vss}%
    \@makecol
    \@outputpage
      \global\vsize\@colroom
    \copy\LT@head\nobreak
  \fi}
\makeatother
% Miscellaneous commands %%%%%%%%%%%%%%%%%%%%%%%%%%%%%%%%%%%%%%%%%%%%%%%%%%%%%%%
\AtBeginDocument{%
\providecommand{\vv}[1]{\overrightarrow{#1}}
\providecommand{\ee}{\mathrm{e}}
\providecolor{alertext}{HTML}{E32518}
\providecommand{\alert}[1]{\textcolor{alertext}{\textbf{#1}}}

\providecommand{\IR}{\mathbb{R}}
\providecommand{\II}{\mathbb{I}}

\renewcommand{\Im}{\operatorname{Im}}
\renewcommand{\Re}{\operatorname{Re}}
}
